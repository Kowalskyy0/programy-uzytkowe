\documentclass[12pt, letterpaper, titlepage]{article}
\usepackage[left=3.5cm, right=2.5cm, top=2.5cm, bottom=2.5cm]{geometry}
\usepackage[MeX]{polski}
\usepackage[utf8]{inputenc}
\usepackage{graphicx}
\usepackage{enumerate}
\usepackage{amsmath} %pakiet matematyczny
\usepackage{amssymb} %pakiet dodatkowych symboli
\usepackage{ulem}
\usepackage{xcolor}
\title{Pierwszy dokument LaTeX}
\author{Mariusz Kowalski}
\date{Październik 2022}


\begin{document}
\maketitle
\begin{enumerate}[I]
\item Punkt 1
\item Punkt 2
\item Punkt 3
\item Punkt 4
\item Punkt 5
\item Punkt 6
\end{enumerate}
\section{tekst}
Tekst dotyczący pierwszej sekcji głównej
\subsection{tekst}
Tekst dotyczący pierwszej podsekcji
\subsubsection{tekst}
Tekst dotyczący pod-podsekcji

\textbf{123}\\
\textit{123}\\
\underline{123}\\
\xout{123123}\\
\uuline{urgent}\\

\clearpage
\begin{center}
{\Huge \textcolor {red} {\textbf{PRZEPIS NA CIASTO}}}
\end{center}

\textcolor {red} {\textbf{Składniki:}}

\begin{enumerate} [-]
\item ½ szklanki mleka lub wody (125ml)
\item 3 kopiaste łyżki kakao
\item 1,5 szklanki cukru (300g)
\item 250g masła lub margaryny
\item 4 jajka
\item 1,5 szklanki mąki pszennej (225g)
\item 2 łyżeczki proszku do pieczenia
\end{enumerate}

\textcolor {red} {\textbf{Sposób przygotowania:}}

\textbf{Mleko} (lub wodę), \textbf{kakao} i \textbf{cukier} przełożyć do garnka i mieszając, zagotować. Do gorącej masy dodać \textbf{masło} i mieszać , aż się rozpuści. Pozostawić do ostygnięcia.
\textbf{Mąkę} wymieszać z proszkiem do pieczenia. Odstawić na bok.
\textbf{Jajka} sparzyć wrzątkiem.  Oddzielić żółtka od białek. Białka ubić na sztywną pianę. Odstawić na bok. Żółtka wmieszać do ostygniętej masy, trzepaczką lub mikserem. Odlać pół szklanki masy. Ostawić na bok. (Będzie to polewa). Trzepaczką (lub mikserem) wmieszać \textbf{mąkę} z \textbf{proszkiem}. Na końcu wmieszać delikatnie szpatułką pianę z białek.
Dno tortownicy o średnicy 26cm wyłożyć papierem do pieczenia, a następnie zacisnąć obręcz. Ciasto przełożyć do formy.
Piec w nagrzanym piekarniku ok. 45 minut, do suchego patyczka, w temperaturze 180°C, grzałka góra- dół. Pozostawić do ostygnięcia.
Ciasto polać odłożoną polewą. (Gdyby polewa była za rzadka, należy włożyć ją na chwilę do lodówki, aż lekko zgęstnieje).

\begin{center}
\includegraphics[width=0.5\textwidth]{ciasto123}
\end{center}

\end{document}
